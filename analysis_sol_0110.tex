\begin{itemize}
\item[1.]
	\begin{itemize}
		\item[(a)] \begin{proof} If $E$ is bounded, $E$ is pre-compact since $\bbr$ is finite (one) dimensional. If $f(E)$ is unbounded, then there exists $\{x_n\} \subseteq E$ such that $f(x_n) \to \infty$. Since $E$ is precompact $\{x_n\}$ has a convergent subsequence, say $\{x_{n_k}\}$ with limit $x \in \bbr$. Then, since $f$ is continuous,
		\[ f(x) = \lim_{k \to \infty} f(x_{n_k}) = \infty \]
	However, since $f$ maps $\bbr$ to $\bbr$, $f(x)$ cannot be $\infty$. 
		\end{proof}
		
		\item[(b)]
		\begin{proof}
		Since $f$ is uniformly continuous, there exists $\delta >0$ such that whenever $|x-y|<\delta$, 
		\[ |f(x)-f(y)| <1 \]
		Since $E$  bounded, it can be covered by finitely many balls of radius $\delta$, say $\{B(x_i,\delta)\}_{i=1}^N$. Then,
		\[ f(E) = \cup_{i=1}^N f(B(x_i,\delta)) \]
		Fix $i$, for any $f(y) \in f(B(x_i,\delta))$,
			\[ |f(y)-f(x_i)| \le 1 \]
		So $f(B(x_i,\delta))$ is bounded. Then, a finite union of bounded sets is also bounded.
		\end{proof}
	\end{itemize}
Counterexample: $E=(0,1)$ and $f(x)=1/x$. $f(E) = (1,\infty)$.

\item[3.]
	\begin{itemize}
		\item[(a)]\begin{proof} Recall the Bessel inequality for any orthonormal set $\{e_n\}$ in an inner product space, $X$. For and $f \in X$,
		\[ \sum |\lip f,e_n\rip|^2 \le \|f\|^2 \]
		In particular, $\lip f,e_n\rip \to 0$ as $n \to \infty$ for any $f \in X$. Now, since 
		\[ \left\{ \dfrac{1}{\sqrt{2\pi}},\dfrac{\cos(n x)}{\sqrt{2\pi}},\dfrac{\sin(n x)}{\sqrt{2\pi}}\right\} \]
		form an orthonormal set in $C[-\pi,\pi]$, we have
		\[ \int_{-\pi}^\pi \sin(2 n x) f(x) \, dx \to 0 \quad \mbox{as } n \to \infty\]
		for any $f \in C[-\pi,\pi]$. Then,
		\[ \int_{-\pi}^\pi \sin^2(nx) f(x) \,dx = \dfrac 12 \int_{-\pi}^\pi f(x) \, dx - \dfrac 12 \int_{-\pi}^\pi \sin(2 n x)f(x) \, dx \to \dfrac 12 \int_{-\pi}^\pi f(x) \, dx \]
		\end{proof}

		\item[(b)]
		\begin{proof}For any $f \in C[-\pi,\pi]$, $n \in \bbn$,
		\begin{align*}
			\left| \int_{-\pi}^\pi \dfrac{x^n}{\pi^n} f(x)\, dx \right|^2 &\le \int_{-\pi}^\pi \dfrac{x^{2n}}{\pi^{2n}} \, dx \int_{-\pi}^\pi |f(x)|^2 \, dx \\
			&= \dfrac{\pi^{2n+1}-(-\pi)^{2n+1}}{(2n+1)\pi^{2n}} \|f\|_{L^2}^2 \\
			&= \dfrac{2\pi}{2n+1} \|f\|_{L^2}^2
		\end{align*}
		which goes to $0$ as $n \to \infty$.
		\end{proof}
	\end{itemize}
    
\item[9.]
    \begin{itemize}
        \item[(a)] There exists $\ep_n \searrow 0$ such that
            \[ \mu\{ |f-g| \ge \ep_n\} \le \ep_n \]
        Then,
            \[ \mu\{f \ne g\} = \mu\{|f-g| >0\} = \mu \left(\bigcup_n \{ |f-g| \ge \ep_n\} \right) \]
            \[= \lim_{n \to \infty} \mu\{|f-g| \le \ep_n\} \le \lim_{n \to \infty} \ep_n =0 \]
        \item[(b)] We only need to show the triangle inequality. Let $t,s >0$, $f,g,h$ measurable functions.
            
        If $|f-h| \le t$ and $|g-h| \le s$, then
	        \[ |f-g| \le |f-h|+|g-h| \le t+s \]
        Thus, $\{|f-h| \le t\} \cap \{|g-h| \le s\} \subset \{|f-g| \le t+s\}$. Then, taking complements, we have
	        \[ \{|f-h| > t\} \cup \{|g-h| > s\} \supset \{ |f-g| > t+s\} \]
        Therefore, $\mu\{|f-g| > t+s\} \le \mu\{|f-h| >t\} + \mu\{|g-h| >s\}$. Let $\delta>0$. There exists $\ep_1,\ep_2$ such that
            \[\mu\{|f-h| > \ep_1\} < \ep_1 \mbox{ and } \ep_1 < \rho(f,h)+\delta/2 \]
        and similarly for $\ep_2$ and $|g-h|$. Therefore,
            \[ \mu\{|f-g| > \ep_1+\ep_2\}\le \mu\{|f-h| >\ep_1\} + \mu\{|g-h| >\ep_2\} < \ep_1+\ep_2 \]
        So,
            \begin{align*} \rho(f,g) &= \inf\{ \ep : \mu\{|f-g| >\ep\} < \ep \} \\
            &\le \ep_1+\ep_2 \\
            &\le \rho(f,h)+\rho(g,h) +\delta
            \end{align*}
        for any $\delta>0$. This proves the Triangle Ineqaulity.
    \end{itemize}
\end{itemize}