
\begin{itemize}
	\begin{center}
		\Large{\textbf{Analysis Prelim Solution - 2018 Summer}}
		\normalsize{\\Yiran Zhu | Clemson - Math}
	\end{center}

	\item[1.] Let $\{a_n\}_{n=1}^\infty$ be a real sequence with $a_n \rightarrow 0$, $n \rightarrow \infty$. Prove that
	$$
	\lim_{N\rightarrow \infty} \frac{1}{N}\sum_{n=1}^Na_n = 0
	$$
 
	\begin{proof}
		For $\epsilon > 0$,  there exists a $M\ge 1$ such that $\forall n\ge M, ~~|a_n| \le \epsilon/2$. For $N\ge M+1$, we have 
		$$
		\begin{aligned}
		 \left\lvert\frac{1}{N}\sum_{n=1}^Na_n\right\rvert &=  \left\lvert\frac{1}{N}\sum_{n=1}^Ma_n + \frac{1}{N} \sum_{n=M+1}^Na_n\right\rvert \le \frac{1}{N}\left\lvert\sum_{n=1}^Ma_n \right\rvert + \frac{1}{N} \sum_{n=M+1}^N\left\lvert a_n\right\rvert  \\
		 &\le \frac{1}{N}\left\lvert\sum_{n=1}^Ma_n \right\rvert + \left(\frac{N-M}{N}\right)\frac{\epsilon}{2}\le \frac{1}{N}\left\lvert\sum_{n=1}^Ma_n \right\rvert + \frac{\epsilon}{2}
		\end{aligned}
		$$
		Let $\widehat{N}$ be an integer greater than $2\left\lvert\sum_{n=1}^Ma_n \right\rvert/\epsilon$ and $\widehat{M} = \max\{\widehat{N}, M+1\}$
		$$
		\forall N \ge \widehat{M}, ~~\left\lvert\frac{1}{N}\sum_{n=1}^Na_n\right\rvert < \epsilon
		$$
		Equivalently, $\lim_{N\rightarrow \infty} \frac{1}{N}\sum_{n=1}^Na_n = 0$.
	\end{proof}

	\item[2.] Let $X$ be a normed linear space and $\emptyset \neq Y\subset X$ be  a subset with the property that
	$X \backslash Y$ is a linear subspace. Show that $Y$ is dense in $X$.

	\begin{proof}
		 Suppose $Y$ is not dense in $X$. Then there exists a point $z\in X\backslash Y$ and a number $r > 0$ such that 
		 $B(z, r) \cap Y=\emptyset$. Equivalently, $B(z,r)\subseteq X\backslash Y$. Then we will show this implies $Y=\emptyset$. Pick $x\in X$ and let $d= \lVert x-z\rVert$. Then
		 $$
		 r>\left\lVert \frac{r(x-z)}{2d}  \right\rVert = \left\lVert \frac{rx- (r-2d)z}{2d} - z \right\rVert ~\Rightarrow~ a:=\frac{rx- (r-2d)z}{2d}\in B(z,r) \subseteq X\backslash Y
		 $$  
		 Since $X\backslash Y$ is a subspace, we have $x = (2da+(r-2d)z)/r \in X\backslash Y$. Note that $x$ is arbitrarily picked from $X$. Therefore, $X\subseteq X\backslash Y \subseteq X ~\Rightarrow~ Y = \emptyset$. By contradiction, $Y$ is dense in $X$.
	\end{proof}
	
	\item[3.] Define $d : \mathbb{R} \times \mathbb{R} \rightarrow \mathbb{R}^+$ by $d(x, y) = |f(x) -f(y)|$ where $f$ is defined as
	$$
	f(x) = \frac{x}{1+|x|}~
	, \forall x\in \mathbb{R}
	$$
	Show that $d$ is a metric on $\mathbb{R}$ and determine if $(\mathbb{R}, d)$ is complete.

	\begin{proof}
		\begin{enumerate}[(i)]
			\item Positive-definite: $d(x,y) = |f(x) - f(y)| \ge 0$ and $d(x,y) = 0 \Leftrightarrow f(x) = f(y)$
			Note that
			$$
			f(x) = f(y) \Leftrightarrow \frac{x}{1+|x|} = \frac{y}{1+|y|} \Leftrightarrow x(1+|y|) = y(1+|x|)
			$$
			Suppose $x< 0$, then $y<0$ and $x(1+|y|) = y(1+|x|) \Rightarrow x = y$. Similarly, if $x\ge 0$, then $y\ge 0$ and $x(1+|y|) = y(1+|x|)\Rightarrow x =y$. In a word, $d(x,y) = 0 \Leftrightarrow x=y$.
			\item Symmetric: $d(x,y) = |f(x)-f(y)| = |f(y)-f(x)| = d(y,x)$
			\item Triangle inequality: $d(x, z)\le d(x,y)+ d(y,z)$ follows from 
			$$|f(x)-f(z)| = |f(x)-f(y)+f(y)-f(z)| \le |f(x)-f(y)| + |f(y)-f(z)|$$ 
		\end{enumerate}
		So $d$ is a metric. However, $(\mathbb{R}, d)$ is not complete. Consider sequence $\{x_n\}_n$ with $x_n = n$. 
		$$
		d(x_n, x_{n+m}) = \left\lvert \frac{m}{(1+n)(1+n+m)} \right\rvert \le \frac{1}{n+1}, ~~\forall n\ge 1, ~\forall m\ge 0  
		$$
		Therefore, $\{x_n\}_n$ is Cauchy. It's obvious that $\{x_n\}_n$ does not converge in $\mathbb{R}$.
	\end{proof}
	
	\item[4.] Let $H$ be a Hilbert space and $Y_1, Y_2$ be two closed linear subspaces in $H$. Denote $P_1$
	and $P_2$ as the orthogonal projections onto $Y_1$ and $Y_2$, respectively. Show that $\lVert P_1 -P_2\rVert \le  1$.
	\begin{proof}
		Observe that $(2P-I)^2 = 4P^2+I-4P = I$ holds for all projection $P$. In particular, if $P$ is orthogonal, then, for all $h\in H$,
		$$
		\lVert (2P-I)h\rVert^2  =\langle (2P-I)h, (2P-I)h\rangle = \langle h, (2P-I)^2h\rangle =\langle h, h\rangle = \lVert h\rVert^2 
		$$
		Therefore, $\lVert 2P - I\rVert = 1 \Rightarrow \lVert P -\frac{1}{2}I\rVert = \frac{1}{2}$. 
		$$
		\lVert P_1 - P_2\rVert \le \left\lVert P_1 -\frac{1}{2}I \right\rVert + \left\lVert P_2 - \frac{1}{2}I\right\rVert \le 1
		$$
	\end{proof}

	\item[5.] Assume $C[0, 1]$ is equipped with the supremum norm and let $T_n : C[0, 1] \rightarrow C[0, 1]$ be defined by
	$$
	T_n(f) = f\left( x^{1+\frac{1}{n}}\right), ~~\forall n\in \mathbb{N}
	$$
	\begin{enumerate}[(a)]
		\item Show that $T_n(f) \rightarrow f, n \rightarrow \infty, \forall f\in C[0,1]$
		\begin{proof}
			Fix $f\in C[0,1]$. Since $[0,1]$ is compact, $f$ is also uniformly continuous on $[0,1]$, i.e.
			for $\epsilon > 0$, there exists $\delta > 0$ such that 
			$$
			\forall x, y\in [0,1] ~s.t~ |x-y| < \delta ~~\Rightarrow~~  |f(x) - f(y) | < \epsilon
			$$
			Let's give an estimation for $g_n(x) := \left\lvert x^{1+\frac{1}{n}} - x \right\rvert = x\left(1 - x^{\frac{1}{n}}\right)$. Obviously, $g_n(x)$ is continuous on $[0,1]$ and $g_n(0) = g_n(1) = 0$. To find the maximum value of $g_n(x)$. we let
			$$
			g'_n(x) = \left(1+\frac{1}{n}\right)x^{\frac{1}{n}} - 1 =0 ~\Rightarrow~ \sup_{x\in [0,1]} g_n(x) = \left( \frac{n}{n+1} \right)^n \frac{1}{n+1} \le \frac{1}{n+1}
			$$
			Pick $N\in \mathbb{Z}^+$ such that $\frac{1}{N+1} < \delta$, then
			$$
			\forall n\ge N, ~\forall x\in [0,1]~~ \left\lvert x^{1+\frac{1}{n}} - x \right\rvert <\frac{1}{N+1}< \delta ~\Rightarrow~ \lVert T_n(f)-f\rVert_\infty < \epsilon 
			$$
			Therefore, $T_n(f)\rightarrow f$ as $n\rightarrow \infty$.
		\end{proof}
		\item For each $n \in N$, find $\lVert T_n - I\rVert$.
		\begin{proof}
			For $f\in C[0,1]$, 
			$$
			\lVert (T_n-I)f\rVert=\lVert T_n(f) - f\rVert_\infty = \sup_{x\in [0,1]} \lvert f(x^{1+\frac{1}{n}}) - f(x)\rvert \le 2 \lVert f\rVert_\infty
			$$
			So $\lVert T_n -I\rVert \le 2$. 
			Let $x_0 = \frac{1}{2}$ and $x_1=  \left(\frac{1}{2}\right)^{1+\frac{1}{n}} \in (0, x_0)$. Construct a function $f$ as follows 
			$$
			f(x)= \left\{
			\begin{aligned}
			&-1& &x\in [0, x_1)\\
			&-1+\frac{2(x-x_1)}{x_0-x_1}& &x\in [x_1, x_0]\\
			&1& &x\in (x_0, 1]\\
			\end{aligned}
			\right.
			$$
			Note that $f(x_0) = 1$ and $f(x_1) = -1$. So $f$ is continuous and $\lVert f\rVert_\infty = 1$. 
			$$
			\lvert (T_n(f)-f)(x_0)\rvert = \lvert f(x_1) - f(x_0)\rvert = 2 = 2\lVert f\rVert_\infty
			$$
			As shown before, $\lVert T_n(f) - f\rVert_\infty \le 2\lVert f\rVert_\infty$. Thus $$\lVert T_n(f) - f\rVert_\infty = 2\lVert f\rVert_\infty ~\Rightarrow~ \lVert T_n-I\rVert = 2$$
		\end{proof}
	\end{enumerate}

	\item[6.] Assume that $\lambda$ is the Lebesgue measure on the real line and $f$ a Lebesgue integrable
	function on the real line. Show that
	$$
	F(x): = \int_{-\infty}^x fd\lambda
	$$
	is uniformly continuous.
	\begin{proof}
		Let $A_n = \{x\in X\mid |f(x)|\ge n\}$. Then, Dominated Convergence Theorem gives
		$$
		\lim_{n\rightarrow \infty} \int_{A_n} |f|d\lambda = \lim_{n\rightarrow \infty} \int_{-\infty}^\infty |f|
		\chi_{A_n}d\lambda = 0 
		$$
		For $\epsilon >0$, there exists $N\ge 1$ such that 
		$$\int_{A_N} |f|d\lambda < \frac{\epsilon}{2}$$
		Then
		$$
		\forall x,y \in \mathbb{R},~~|F(x) - F(y)| = \left|\int_{-\infty}^x fd\lambda - \int_{-\infty}^y fd\lambda\right| = \left\lvert\int_y^x fd\lambda\right\rvert \le \int_y^x \left\lvert f \right\rvert d\lambda
		$$		
		Observe that, if $\lvert x-y\rvert < \frac{\epsilon}{2N}$, then
		$$
		 \int_y^x \left\lvert f \right\rvert d\lambda = \int_{[x,y]\cap A_N} \left\lvert f \right\rvert d\lambda + \int_{[x,y]\backslash A_N} |f| d\lambda \le \int_{A_N} |f| d\lambda + N |x-y| < \epsilon
		$$
	\end{proof}

	\item[7.] Let $(X,\mathcal{M}, \mu)$ be a measure space and $\{A_n\}_n$ be a sequence of sets in $\mathcal{M}$. Recall that
	$\lim \sup_{n\rightarrow \infty} A_n := \cap_{n=1}^\infty \cup_{k=n}^\infty A_k$.
	\begin{enumerate}[(a)]
		\item Prove that if $\sum_{n=1}^\infty
		\mu(A_n) < \infty$, then $\mu(\lim\sup_{n\rightarrow \infty}A_n ) = 0$
		\begin{proof}
			Observe that 
			$$
			\mu\left(\lim\sup_{n\rightarrow \infty}A_n \right)\le \mu\left(\cup_{k=n}^\infty A_k  \right) \le  \sum_{k=n}^\infty \mu(A_k), ~~\forall n\ge 1
			$$
			Since $\sum_{n=1}^\infty \mu(A_n) < \infty$, for $\epsilon > 0$, there exists $N\ge 1$ such that 
			$$
			\sum_{k=N}^\infty \mu(A_k) < \epsilon ~\Rightarrow~ 	\mu\left(\lim\sup_{n\rightarrow \infty}A_n \right)\le \epsilon
			$$
			Let $\epsilon \rightarrow 0$, we derive $	\mu\left(\lim\sup_{n\rightarrow \infty}A_n \right) = 0$.
		\end{proof}
		\item Is the converse true? If yes, prove it. If no, give a counter-example.
		\begin{proof}
			Converse is not true. Consider $A_n = [0, 1/n]$. Then $\cup_{k=n}^\infty A_k = A_n = [0, 1/n]$. 
			$$
			\limsup_{n\rightarrow \infty} A_n = \lim_{N\rightarrow \infty} \cap_{n=1}^N \cup_{k=n}^\infty A_k = \lim_{N\rightarrow \infty} \left[0, \frac{1}{N} \right] = \{0\}
			$$
			Therefore,
			$$
			\mu\left(\lim\sup_{n\rightarrow \infty}A_n \right) = 0
			$$
			However, $\mu(A_n) = \frac{1}{n}$ for all $n\in \mathbb{N}$
			$$
			\sum_{n=1}^\infty \mu(A_n) = \sum_{n=1}^\infty  \frac{1}{n} = \infty
			$$
		\end{proof}
	\end{enumerate}

	\item[8.] Let $(X,\mathcal{M}, \mu)$ be a finite measure space. Prove that a monotone increasing sequence
	of measurable functions $f_n : X \rightarrow \mathbb{R}$ converges in measure if and only if it converges pointwise
	a.e..
	\begin{proof}
		\begin{enumerate}[(i)]
			\item Suppose $f_n$ converges to $f$ poinwise a.e.: By Egorov theorem, for each $\epsilon > 0$, there exists a measurable set $E$ with measure $\mu(E) < \epsilon$ such that $f_n$ converges uniformly to $f$ on $X\backslash E$. In other words, for each $\delta > 0$, there exists $N\ge 1$ such that 
			$$
			\forall n\ge N,~~ \lvert f_n(x) - f(x)\rvert < \delta, ~\forall x\in X\backslash E
			$$
			Consequently,
			$$
			\forall n\ge N,~~A_n:=\{x\in X \mid \lvert f_n(x) - f(x)\rvert \ge \delta\} \subseteq E ~\Rightarrow~ \mu(A_n) < \epsilon
			$$
			Therefore, $f_n$ converges to $f$ in measure.
			
			\item Suppose $f_n$ converges to $f$ in measure: There exists a subsequence $\{f_{n_k}\}_k$ converges to $f$ pointwise a.e.. Let $E$ be the zero-measure set that $\{f_{n_k}\}_k$ does not converge to $f$. Then
			fix $x\in X\backslash E$, for each $\epsilon > 0$, there exists $N\ge 1$ such that 
			$$
			\forall k\ge N,~~ |f_{n_k}(x) - f(x)|  < \epsilon
			$$
			Since $\{f_n\}_n$ is monotone increasing, we have
			$$
			\forall n\ge n_N,~~ |f_{n}(x) - f(x)| = f(x) - f_n(x) \le f(x) - f_{n_N}(x)  = |f_{n_k}(x) - f(x)|< \epsilon
			$$ 
			Note that $\{f_{n_k}\}_k$ is also monotone increasing. 
		\end{enumerate}
	\end{proof}

	\item[9.] Suppose that $g$ is a non-negative Borel measurable function on $\mathbb{R}$ with $\int_{\mathbb{R}} gd\lambda = 1$
	where $\lambda$ denotes Lebesgue measure on $\mathbb{R}$. For $k\in \mathbb{N}$ set $g_k(x) = kg(kx)$. Let $f$ be a
	bounded continuous function. Prove that
	$$
	\lim_{k\rightarrow \infty } \int_{\mathbb{R}} g_kfd\lambda = f(0)
	$$
	\begin{proof}
		Suppose $\sup_{x\in \mathbb{R}}|f(x)| = M$ and define $h_k(x) = g(x)f(x/k)$. Observe that 
		$$
		\int_{\mathbb{R}} g_k fd\lambda = \int_{\mathbb{R}} kg(kx)f(x)d\lambda = \int_{\mathbb{R}} g(x)f(x/k)d\lambda = \int_{\mathbb{R}} h_k d\lambda
		$$
		In order to apply Dominated Convergence Theorem, we need to show $h_k$ is uniformly bounded by a integrable function. Indeed, 
		$$
		\forall k\ge 1,~~|h_k(x)| = |g(x)f(x/k)|\le Mg(x) ~\text{  and  }~ \int_{\mathbb{R}} Mgd\lambda = M < \infty 
		$$
		By DCT,
		$$
		\lim_{k\rightarrow \infty} \int_{\mathbb{R}} h_kd\lambda = \int_{\mathbb{R}} \lim_{k\rightarrow \infty } h_k d\lambda = \int_{\mathbb{R}} f(0)gd\lambda = f(0)
		$$
	\end{proof}
	\item[10.]  Let $\lambda$ be the Lebesgue measure on $(0, 1)$. Suppose the $f_n : (0, 1) \rightarrow [0, \infty)$ is
	a sequence of Borel measurable functions such that $\int_{(0,1)}f_nd\lambda=1$ for all $n \ge 1$ and
	$\lim_{n\rightarrow \infty}f_n(x) = x$ for all $x\in (0, 1)$.
	\begin{enumerate}[(a)]
		\item Give an example of such a sequence.
		\begin{proof}
			$$
			f_n(x)  = \left\{
			\begin{aligned}
			&(n+1)\left( 1 - nx\right)& &x\in \left(0, \frac{1}{n}\right)\\
			&\frac{n}{n-1}\left( x-\frac{1}{n}\right)& &x\in \left[\frac{1}{n}, 1\right)\\
			\end{aligned}\right.
			$$
			Then
			$$
			\int_{(0,1)} f_n d\lambda= \int_{ \left(0, \frac{1}{n}\right)}(n+1)\left( 1 - nx\right)d\lambda + \int_{\left[\frac{1}{n}, 1\right)}\frac{n}{n-1}\left( x-\frac{1}{n}\right) d\lambda = \frac{n+1}{2n} + \frac{n-1}{2n} = 1
			$$
			Fix $x\in (0,1)$, there exists $N\ge 1$ such that $x > \frac{1}{N}$, then
			$$
			\forall n\ge N, ~~f_n(x) -x= \frac{n}{n-1}\left( x-\frac{1}{n}\right)- x= \frac{x-1}{n-1}
			$$
			So $f_n(x) \rightarrow x$ as $n\rightarrow \infty$.
		\end{proof}
		\item Show that one can find $n \ge 1$ and $x \in (0, 1)$ such that  $f_n(x)\sqrt{x} \ge 2018$
		\begin{proof}
			Prove by contradiction. Suppose not, then 
			$$
			\forall n\ge 1,~~ f_n(x)\le \frac{2018}{\sqrt{x}}
			$$
			Note that 
			$$
			\int_{(0,1)} \frac{2018}{\sqrt{x}} d\lambda = 4036 < \infty
			$$
			By Dominated Convergence Theorem, 
			$$
			1 = \lim_{n\rightarrow \infty} \int_{(0,1)} f_n d\lambda = \int_{(0,1)} \lim_{n\rightarrow \infty} f_n d\lambda = \int_{(0,1)} x d\lambda = \frac{1}{2}
			$$
			Above equality cannot be true. 
		\end{proof}
	\end{enumerate}

\end{itemize}